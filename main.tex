%%%%%%%%%%%%%%%%%
% This is an sample CV template created using altacv.cls
% (v1.3, 10 May 2020) written by LianTze Lim (liantze@gmail.com). Now compiles with pdfLaTeX, XeLaTeX and LuaLaTeX.
% This fork/modified version has been made by Nicolás Omar González Passerino (nicolas.passerino@gmail.com, 15 Oct 2020)
%
%% It may be distributed and/or modified under the
%% conditions of the LaTeX Project Public License, either version 1.3
%% of this license or (at your option) any later version.
%% The latest version of this license is in
%%    http://www.latex-project.org/lppl.txt
%% and version 1.3 or later is part of all distributions of LaTeX
%% version 2003/12/01 or later.
%%%%%%%%%%%%%%%%

%% If you need to pass whatever options to xcolor
\PassOptionsToPackage{dvipsnames}{xcolor}

%% If you are using \orcid or academicons
%% icons, make sure you have the academicons
%% option here, and compile with XeLaTeX
%% or LuaLaTeX.
% \documentclass[10pt,a4paper,academicons]{altacv}

%% Use the "normalphoto" option if you want a normal photo instead of cropped to a circle
% \documentclass[10pt,a4paper,normalphoto]{altacv}

\documentclass[10pt,a4paper,ragged2e,withhyper]{altacv}

%% AltaCV uses the fontawesome5 and academicons fonts
%% and packages.
%% See http://texdoc.net/pkg/fontawesome5 and http://texdoc.net/pkg/academicons for full list of symbols. You MUST compile with XeLaTeX or LuaLaTeX if you want to use academicons.

% Change the page layout if you need to
\geometry{left=1.2cm,right=1.2cm,top=1cm,bottom=1cm,columnsep=0.8cm}

% The paracol package lets you typeset columns of text in parallel
\usepackage{paracol}

% Change the font if you want to, depending on whether
% you're using pdflatex or xelatex/lualatex
\ifxetexorluatex
  % If using xelatex or lualatex:
  \setmainfont{Roboto Slab}
  \setsansfont{Lato}
  \renewcommand{\familydefault}{\sfdefault}
\else
  % If using pdflatex:
  \usepackage[rm]{roboto}
  \usepackage[defaultsans]{lato}
  % \usepackage{sourcesanspro}
  \renewcommand{\familydefault}{\sfdefault}
\fi

% ----- LIGHT MODE -----
\definecolor{SlateGrey}{HTML}{2E2E2E}
\definecolor{LightGrey}{HTML}{666666}
\definecolor{PrimaryColor}{HTML}{3498db}
\definecolor{SecondaryColor}{HTML}{53a89e}
% \definecolor{SecondaryColor}{HTML}{e67e22}
\definecolor{ThirdColor}{HTML}{40c3a3}
\definecolor{BackgroundColor}{HTML}{ffffff}
\colorlet{name}{SlateGrey}
\colorlet{tagline}{SlateGrey}
\colorlet{heading}{SlateGrey}
\colorlet{headingrule}{LightGrey}
\colorlet{subheading}{SecondaryColor}
\colorlet{accent}{SecondaryColor}
\colorlet{emphasis}{SlateGrey}
\colorlet{body}{LightGrey}
\pagecolor{BackgroundColor}   
% ----- DARK MODE -----
%\definecolor{BackgroundColor}{HTML}{242424}
%\definecolor{SlateGrey}{HTML}{6F6F6F}
%\definecolor{LightGrey}{HTML}{ABABAB}
%\definecolor{PrimaryColor}{HTML}{3F7FFF}
%\colorlet{name}{PrimaryColor}
%\colorlet{tagline}{PrimaryColor}
%\colorlet{heading}{PrimaryColor}
%\colorlet{headingrule}{PrimaryColor}
%\colorlet{subheading}{PrimaryColor}
%\colorlet{accent}{PrimaryColor}
%\colorlet{emphasis}{LightGrey}
%\colorlet{body}{LightGrey}
%\pagecolor{BackgroundColor}

% Change some fonts, if necessary
\renewcommand{\namefont}{\Huge\rmfamily\bfseries}
\renewcommand{\personalinfofont}{\small\bfseries}
\renewcommand{\cvsectionfont}{\LARGE\rmfamily\bfseries}
\renewcommand{\cvsubsectionfont}{\large\bfseries}

% Change the bullets for itemize and rating marker
% for \cvskill if you want to
\renewcommand{\itemmarker}{{\small\textbullet}}
\renewcommand{\ratingmarker}{\faCircle}

%% sample.bib contains your publications
%% \addbibresource{sample.bib}
\usepackage{hyperref}
\hypersetup{
    colorlinks=true,
    linkcolor=blue,
    }
\usepackage{url}
\urlstyle{same}
    
\begin{document}
    \name{Carlos Victor Dantas Araújo}%% You can add multiple photos on the left or right

    \personalinfo{
        \email{carlosvdaraujo@gmail.com}\smallskip
        \phone{(88) 99406-8577} \smallskip
        \printinfo{\faLinkedin}{Carlos Araújo}[https://linkedin.com/in/carlos-araújo-1878b0121]
        \printinfo{\faAddressBook}{WebPage}[https://www.ic.unicamp.br/~ra230261/]
        
   
        %\homepage{nicolasomar.me}
        %\medium{nicolasomar}
        %% You MUST add the academicons option to \documentclass, then compile with LuaLaTeX or XeLaTeX, if you want to use \orcid or other academicons commands.
        % \orcid{0000-0000-0000-0000}
        %% You can add your own arbtrary detail with
        %% \printinfo{symbol}{detail}[optional hyperlink prefix]
        % \printinfo{\faPaw}{Hey ho!}[https://example.com/]
        %% Or you can declare your own field with
        %% \NewInfoFiled{fieldname}{symbol}[optional hyperlink prefix] and use it:
        % \NewInfoField{gitlab}{\faGitlab}[https://gitlab.com/]
        % \gitlab{your_id}
    }
    
    \makecvheader
    %% Depending on your tastes, you may want to make fonts of itemize environments slightly smaller
    % \AtBeginEnvironment{itemize}{\small}
    
    %% Set the left/right column width ratio to 6:4.
    \columnratio{0.29}

    % Start a 2-column paracol. Both the left and right columns will automatically
    % break across pages if things get too long.
    \begin{paracol}{2}
        % ----- STRENGTHS -----
        \cvsection{Soft Skills}
            \cvtag{Communication}
            \cvtag{Adaptability}
            \cvtag{Teamwork}
            \cvtag{Decision-making}
            \cvtag{Problem-solving}
            \medskip
        % ----- STRENGTHS -----
        
        % ----- LEARNING -----
        \cvsection{Hard Skills}
            \cvtag{C/C++}
            \cvtag{Java}
            \cvtag{Python}
            \cvtag{R}
            \cvtag{Go}
            \cvtag{LaTeX}
            \cvtag{Git}
            \cvtag{CI/CD}
            \cvtag{GAMA}
            \cvtag{NS-2}
            \cvtag{SUMO}
            \linebreak
            \cvtag{Stochastic Optimization}
            \linebreak
            \cvtag{Heuristics}
            \cvtag{Simheuristics}
            \linebreak
            \cvtag{Mathematical Models}
            \linebreak
            \cvtag{Operations Research}
            \linebreak
            \cvtag{Combinatorial Optimization}
            \linebreak
            \cvtag{Statistical Analysis}
            \linebreak
            \cvtag{Simulation}
            
        % ----- LEARNING -----
        \cvsection{Libraries}
            \cvtag{Cplex}
            \cvtag{Gurobi}
            \cvtag{AMPL}
            \cvtag{Boost}
            \cvtag{Lemon}
            \cvtag{Pandas}
            \cvtag{Numpy}
            \cvtag{Matplotlib}
            \cvtag{Flask}
            \cvtag{scmamp}
            
       
        % ----- LEARNING -----
        
        % ----- LANGUAGES -----
        \cvsection{Languages}
            \cvlang{Portuguese}{Native}\\
            \cvlang{English}{Advanced}
            %% Yeah I didn't spend too much time making all the
            %% spacing consistent... sorry. Use \smallskip, \medskip,
            %% \bigskip, \vpsace etc to make ajustments.
            \smallskip
        % ----- LANGUAGES -----
        
        % ----- MOST PROUD -----
        % \cvsection{Most Proud of}
        
        % \cvachievement{\faTrophy}{Fantastic Achievement}{and some details about it}\\
        % \divider
        % \cvachievement{\faHeartbeat}{Another achievement}{more details about it of course}\\
        % \divider
        % \cvachievement{\faHeartbeat}{Another achievement}{more details about it of course}
        % ----- MOST PROUD -----
        
        % \cvsection{A Day of My Life}
        
        % Adapted from @Jake's answer from http://tex.stackexchange.com/a/82729/226
        % \wheelchart{outer radius}{inner radius}{
        % comma-separated list of value/text width/color/detail}
        % \wheelchart{1.5cm}{0.5cm}{%
        %   6/8em/accent!30/{Sleep,\\beautiful sleep},
        %   3/8em/accent!40/Hopeful novelist by night,
        %   8/8em/accent!60/Daytime job,
        %   2/10em/accent/Sports and relaxation,
        %   5/6em/accent!20/Spending time with family
        % }
        
        % use ONLY \newpage if you want to force a page break for
        % ONLY the current column
        \newpage
        
        %% Switch to the right column. This will now automatically move to the second
        %% page if the content is too long.
        \switchcolumn
        
        % ----- EXPERIENCE -----
        \cvsection{Work Experience}
        \cvevent{Senior Optimization Analyst}{\textbf{| KaBuM!}}{Jan. 2022 - Actual}{Limeira - SP}
            {Researcher and Developer specialized in Warehouse and Routing}. \\
            \medskip
            \begin{itemize}
                \item Involved in the design, management, and development of
                  multiple optimization projects.
                \item Development of mathematical models, exact, and heuristic solutions for grouping, packing, and routing problems.
            \end{itemize}
        \divider
        \cvevent{AI Researcher}{\textbf{| I.Systems}}{Apr. 2021 - Dec. 2021}{Campinas - SP}
            {Researcher and developer specialized in Supply Chain, more specifically in production
              planning and job scheduling}.\\
            \medskip
            \begin{itemize}
                \item Involved in the main heuristics development.
                \item Deploy of the solution in various cloud environments.
            \end{itemize}
        % ----- Pesquisa -----
        \cvsection{Research Experience}
        \cvevent{Post-graduate}{\textbf{| \href{http://www.loco.ic.unicamp.br/}{Laboratory of Optimization and Combinatorics}}}{Mar. 2019 - Actual}{Campinas - SP}{}
	      \begin{itemize}
	            \item Study of stochastic optimization and development of simheuristic.
	            \item Development of simulation-based instances for Arc Routing and Multicast Routing Problems. 
		        \item Formulation and development of Lagrangian relaxations, (meta)heuristics, and hybrid approaches.
		  \end{itemize}
		\divider
		\medskip
		\cvevent{Scientific Initiation}{\textbf{| \href{http://www.nemo.ufc.br/}{NEMo}}}{Feb. 2017 - Dec. 2020}{Russas - CE}{}
        \begin{itemize}
        	\item Formulation and development of relaxations and heuristics for the Maximally Diverse Grouping Problem.
        	\item polyhedral study and development of heuristics using optimality cuts for the Max Cut Problem.
        	\item Study of Data Science and Machine Learning algorithms, generating results applied in Kaggle competitions.
        \end{itemize}
        % ----- EXPERIENCE -----
        \cvsection{Teaching Experience}
        \medskip
        \cvevent{Assistant Professor}{\textbf{| University of Campinas}}{Jan. 2020 - Dec. 2020}{Campinas - SP}
        {Assistant Professor of disciplines Programming Challenges I - MO521 and
          Introduction to Programming and Algorithms - MO102. \\}
        \divider
        \medskip
        \cvevent{Assistant Professor}{\textbf{| University of Ceará}}{Jan. 2017 - Dec. 2017}{Russas - CE}
        {Assistant Professor in the discipline of Introduction to Programming. \\}
        \divider
    \end{paracol}
    
    \begin{paracol}{1}
    % ----- EDUCATION -----
        \cvsection{Education}
            \medskip
            \cvevent{\textbf{Ph.D. in Computer Science}}{| Combinatorial Optimization}{Mar. 2021 - Mar. 2025}{University of Campinas}
            \medskip
            \begin{itemize}
            \item{GPA: 4.0 on a scale of 4.0}
			      \item{Advisors: Professor Ph.D. Fábio L. Usberti and Ph.D. Rafael K. Arakaki}
			      \item{Courses: \textcolor{SecondaryColor}{Parallel Programming},
                \textcolor{SecondaryColor}{Algorithms and Complexity} and
                \textcolor{SecondaryColor}{Approximation Algorithms}.}
		      \end{itemize}
            \divider
            \cvevent{\textbf{MSc. in Computer Science}}{| Combinatorial Optimization}{Mar. 2019 - Mar. 2021}{University of Campinas}
            \medskip
            \begin{itemize}
            \item{GPA: 3.6 on a scale of 4.0}
			      \item{Advisors: Professor Ph.D. Fábio L. Usberti and Professor Ph.D. Cid C. de Souza}
			      \item{Dissertation: Formulation and Heuristics for the problem of
                Maximum Service in Multicast Routing with QoS Constraints -
                a Portuguese version is available in this \href{https://hdl.handle.net/20.500.12733/1641777}{\textcolor{ThirdColor}{link}}}
			      \item{Courses: \textcolor{SecondaryColor}{Algorithms in Graphs},
                \textcolor{SecondaryColor}{Integer and Linear Programming} and
                \textcolor{SecondaryColor}{Combinatorial Optimization Topics}.}
		      \end{itemize}
            \divider
            \cvevent{\textbf{B.S. in Computer Science}}{}{Mar. 2015 - Dec. 2018}{University of Ceará}
            % \medskip
            \begin{itemize}
            \item{GPA: 8.46 on a scale of 10.0}
          \item{Advisor: Professor Ph.D. Pablo L. B. Soares}
          \item{Conclusion Work: Utilização de desigualdades válidas baseadas
                em condições de otimalidade na construção de algoritmos
                heurísticos para o problema do corte máximo - a Portuguese
                version is available in this \href{http://www.repositorio.ufc.br/handle/riufc/39085}{\textcolor{ThirdColor}{link}}}
		      \end{itemize}
		      
        % ----- EDUCATION -----
        \cvsection{Published Works}
        \medskip
        \textbf{Araújo, C. V. D.}; Andrade, M. D.; Usberti, F. L.; Arakaki, R. K.
        \textcolor{SecondaryColor}{A Prize-Collecting Approach for the Dengue
          Arc Routing Problem}. Simpósio Brasileiro de Pesquisa Operacional
        (SBPO), 2022. Vol. 0 p. 0-0 (\emph{Approved}) \\
        \divider \\ 
        \textbf{Araújo, C. V. D.}; Usberti, F. L.; de Souza, C. C.
        \textcolor{SecondaryColor}{Lagrangian Relaxation for the Problem of Maximum Service in Multicast Routing with QoS constraints}. International Transactions in Operational Research (ITOR), 2022. Vol. 0 p. 0-0 \\
        \divider \\
        \textbf{Araújo, C. V. D.}; Figueiredo, T. F.
        \textcolor{SecondaryColor}{O Problema Da Diversidade Máxima de Grupos: uma abordagem de programação linear inteira}.  L Simpósio Brasileiro de Pesquisa Operacional (SBPO), 2018. \\
    		\divider \\
    		\textbf{Araújo, C. V. D.}; Figueiredo, T. F.
    		\textcolor{SecondaryColor}{Relaxação Lagrangiana Aplicada ao Problema da Diversidade Máxima de Grupos}. Encontros universitários - UFC, 2018. Vol. 0 p. 0-0 (\emph{}) \\
    		\divider \\
    		\textbf{Araújo, C. V. D.}; Soares, P. L. B. 
    		\textcolor{SecondaryColor}{Algoritmo Genético para o Problema Do Corte Máximo}. Encontros universitários - UFC, 2018. Vol. 0 p. 0-0 (\emph{Resumo}) \\
            \divider \\
            \textbf{Araújo, C. V. D.}; Soares, P. L. B.
            \textcolor{SecondaryColor}{Estudo de Abordagens para o problema de Corte Máximo}. Encontros universitários  - UFC, 2017.Vol. 0 p. 0-0 (\emph{Resumo}) \\
	    \medskip
        % ----- PROJECTS -----
      \cvsection{Professional honors, Awards and Fellowships}
        \begin{itemize}
            \item Best Scientific Initiation work in the
              \textcolor{SecondaryColor}{UFC University Meetings, 2018}.
            \item Second place in the regional phase of the \textcolor{SecondaryColor}{\textit{International Collegiate Programming Contest} (ICPC), 2018}.
            \item National winner of the \textcolor{SecondaryColor}{Portuguese
                Language Olympiad - Escrevendo o Futuro (2008)}.     
        \end{itemize}
    \end{paracol}
\end{document}
