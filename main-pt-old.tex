%%%%%%%%%%%%%%%%%
% This is an sample CV template created using altacv.cls
% (v1.3, 10 May 2020) written by LianTze Lim (liantze@gmail.com). Now compiles with pdfLaTeX, XeLaTeX and LuaLaTeX.
% This fork/modified version has been made by Nicolás Omar González Passerino (nicolas.passerino@gmail.com, 15 Oct 2020)
%
%% It may be distributed and/or modified under the
%% conditions of the LaTeX Project Public License, either version 1.3
%% of this license or (at your option) any later version.
%% The latest version of this license is in
%%    http://www.latex-project.org/lppl.txt
%% and version 1.3 or later is part of all distributions of LaTeX
%% version 2003/12/01 or later.
%%%%%%%%%%%%%%%%

%% If you need to pass whatever options to xcolor
\PassOptionsToPackage{dvipsnames}{xcolor}

%% If you are using \orcid or academicons
%% icons, make sure you have the academicons
%% option here, and compile with XeLaTeX
%% or LuaLaTeX.
% \documentclass[10pt,a4paper,academicons]{altacv}

%% Use the "normalphoto" option if you want a normal photo instead of cropped to a circle
% \documentclass[10pt,a4paper,normalphoto]{altacv}

\documentclass[10pt,a4paper,ragged2e,withhyper]{altacv}

%% AltaCV uses the fontawesome5 and academicons fonts
%% and packages.
%% See http://texdoc.net/pkg/fontawesome5 and http://texdoc.net/pkg/academicons for full list of symbols. You MUST compile with XeLaTeX or LuaLaTeX if you want to use academicons.

% Change the page layout if you need to
\geometry{left=1.2cm,right=1.2cm,top=1cm,bottom=1cm,columnsep=0.8cm}

% The paracol package lets you typeset columns of text in parallel
\usepackage{paracol}

% Change the font if you want to, depending on whether
% you're using pdflatex or xelatex/lualatex
\ifxetexorluatex
  % If using xelatex or lualatex:
  \setmainfont{Roboto Slab}
  \setsansfont{Lato}
  \renewcommand{\familydefault}{\sfdefault}
\else
  % If using pdflatex:
  \usepackage[rm]{roboto}
  \usepackage[defaultsans]{lato}
  % \usepackage{sourcesanspro}
  \renewcommand{\familydefault}{\sfdefault}
\fi

% ----- LIGHT MODE -----
\definecolor{SlateGrey}{HTML}{2E2E2E}
\definecolor{LightGrey}{HTML}{666666}
\definecolor{PrimaryColor}{HTML}{3498db}
\definecolor{SecondaryColor}{HTML}{53a89e}
% \definecolor{SecondaryColor}{HTML}{e67e22}
\definecolor{ThirdColor}{HTML}{40c3a3}
\definecolor{BackgroundColor}{HTML}{ffffff}
\colorlet{name}{SlateGrey}
\colorlet{tagline}{SlateGrey}
\colorlet{heading}{SlateGrey}
\colorlet{headingrule}{LightGrey}
\colorlet{subheading}{SecondaryColor}
\colorlet{accent}{SecondaryColor}
\colorlet{emphasis}{SlateGrey}
\colorlet{body}{LightGrey}
\pagecolor{BackgroundColor}   
% ----- DARK MODE -----
%\definecolor{BackgroundColor}{HTML}{242424}
%\definecolor{SlateGrey}{HTML}{6F6F6F}
%\definecolor{LightGrey}{HTML}{ABABAB}
%\definecolor{PrimaryColor}{HTML}{3F7FFF}
%\colorlet{name}{PrimaryColor}
%\colorlet{tagline}{PrimaryColor}
%\colorlet{heading}{PrimaryColor}
%\colorlet{headingrule}{PrimaryColor}
%\colorlet{subheading}{PrimaryColor}
%\colorlet{accent}{PrimaryColor}
%\colorlet{emphasis}{LightGrey}
%\colorlet{body}{LightGrey}
%\pagecolor{BackgroundColor}

% Change some fonts, if necessary
\renewcommand{\namefont}{\Huge\rmfamily\bfseries}
\renewcommand{\personalinfofont}{\small\bfseries}
\renewcommand{\cvsectionfont}{\LARGE\rmfamily\bfseries}
\renewcommand{\cvsubsectionfont}{\large\bfseries}

% Change the bullets for itemize and rating marker
% for \cvskill if you want to
\renewcommand{\itemmarker}{{\small\textbullet}}
\renewcommand{\ratingmarker}{\faCircle}

%% sample.bib contains your publications
%% \addbibresource{sample.bib}
\usepackage{hyperref}
\hypersetup{
    colorlinks=true,
    linkcolor=blue,
    }
\usepackage{url}
\urlstyle{same}
    
\begin{document}
    \name{Carlos Victor Dantas Araújo}%% You can add multiple photos on the left or right

    \personalinfo{
        \email{carlosvdaraujo@gmail.com}\smallskip
        \phone{(88) 99406-8577} \smallskip
        \printinfo{\faLinkedin}{Carlos Araújo}[https://linkedin.com/in/carlos-araújo-1878b0121]
   
        %\homepage{nicolasomar.me}
        %\medium{nicolasomar}
        %% You MUST add the academicons option to \documentclass, then compile with LuaLaTeX or XeLaTeX, if you want to use \orcid or other academicons commands.
        % \orcid{0000-0000-0000-0000}
        %% You can add your own arbtrary detail with
        %% \printinfo{symbol}{detail}[optional hyperlink prefix]
        % \printinfo{\faPaw}{Hey ho!}[https://example.com/]
        %% Or you can declare your own field with
        %% \NewInfoFiled{fieldname}{symbol}[optional hyperlink prefix] and use it:
        % \NewInfoField{gitlab}{\faGitlab}[https://gitlab.com/]
        % \gitlab{your_id}
    }
    
    \makecvheader
    %% Depending on your tastes, you may want to make fonts of itemize environments slightly smaller
    % \AtBeginEnvironment{itemize}{\small}
    
    %% Set the left/right column width ratio to 6:4.
    \columnratio{0.29}

    % Start a 2-column paracol. Both the left and right columns will automatically
    % break across pages if things get too long.
    \begin{paracol}{2}
        % ----- STRENGTHS -----
        \cvsection{Soft Skills}
            \cvtag{Comunicação}
            \cvtag{Proatividade}
            \cvtag{Colaboração}
            \cvtag{Flexibilidade}
            \cvtag{Orientação por Resultados}
            \medskip
        % ----- STRENGTHS -----
        
        % ----- LEARNING -----
        \cvsection{Hard Skills}
            \cvtag{C++}
            \cvtag{Java}
            \cvtag{Python}
            \cvtag{R}
            \cvtag{Go}
            \cvtag{LaTeX}
            \cvtag{Git}
            \linebreak
            \cvtag{(meta)Heurísticas}
            \linebreak
            \cvtag{Modelos Matemáticos}
            \linebreak
            \cvtag{Pesquisa Operacional}
            \linebreak
            \cvtag{Análise Estatística}
            
        % ----- LEARNING -----
        \cvsection{Bibliotecas}
            \cvtag{ILOG CPLEX}
            \cvtag{Gurobi}
            \cvtag{AMPL}
            \cvtag{Boost Library}
            \cvtag{Lemon Graph}
            \cvtag{Pandas}
            \cvtag{Numpy}
            \cvtag{Matplotlib}
            \cvtag{Flask}
            \cvtag{scmamp}
            
       
        % ----- LEARNING -----
        
        % ----- LANGUAGES -----
        \cvsection{Línguas}
            \cvlang{Português}{Nativo}\\
            \cvlang{Inglês}{Intermediário}
            %% Yeah I didn't spend too much time making all the
            %% spacing consistent... sorry. Use \smallskip, \medskip,
            %% \bigskip, \vpsace etc to make ajustments.
            \smallskip
        % ----- LANGUAGES -----
        
        % ----- MOST PROUD -----
        % \cvsection{Most Proud of}
        
        % \cvachievement{\faTrophy}{Fantastic Achievement}{and some details about it}\\
        % \divider
        % \cvachievement{\faHeartbeat}{Another achievement}{more details about it of course}\\
        % \divider
        % \cvachievement{\faHeartbeat}{Another achievement}{more details about it of course}
        % ----- MOST PROUD -----
        
        % \cvsection{A Day of My Life}
        
        % Adapted from @Jake's answer from http://tex.stackexchange.com/a/82729/226
        % \wheelchart{outer radius}{inner radius}{
        % comma-separated list of value/text width/color/detail}
        % \wheelchart{1.5cm}{0.5cm}{%
        %   6/8em/accent!30/{Sleep,\\beautiful sleep},
        %   3/8em/accent!40/Hopeful novelist by night,
        %   8/8em/accent!60/Daytime job,
        %   2/10em/accent/Sports and relaxation,
        %   5/6em/accent!20/Spending time with family
        % }
        
        % use ONLY \newpage if you want to force a page break for
        % ONLY the current column
        \newpage
        
        %% Switch to the right column. This will now automatically move to the second
        %% page if the content is too long.
        \switchcolumn
        
        % ----- EXPERIENCE -----
        \cvsection{Experiência de Mercado}
        \cvevent{Analista de Otimização Sênior}{\textbf{| KaBuM!}}{Jan. 2022 - Atual}{Limeira - SP}
            {Pesquisador e Desenvolvedor focado em problemas de Centro de Distribuição e roteamento}. \\
            \medskip
            \textbf{Principais responsabilidades:}
            \medskip
            \begin{itemize}
                \item Envolvido ativamente na idealização, gerência e desenvolvimento dos projetos de otimização.
                \item Desenvolvimento de soluções exatas e heurísticas para problemas agrupamento, empacotamento e roteamento. 
            \end{itemize}
        \divider
        \cvevent{Pesquisador IA}{\textbf{| I.Systems - Automação Avançada de Processos}}{Abr. 2021 - Dez. 2021}{Campinas - SP}
            {Pesquisador em Otimização Combinatória focado em \textit{Supply Chain}, mais especificamente para planejamento de Produção e escalonamento de tarefas.}.\\
            \medskip
            \textbf{Principais responsabilidades:}
            \medskip
            \begin{itemize}
                \item Envolvido ativamente na idealização e implementação das heurísticas principais do algoritmo.
                \item Participação ativa da preparação do ambiente \textit{cloud} para execução do algoritmo no sistema web.
            \end{itemize}
        % ----- Pesquisa -----
        \cvsection{Experiência com Pesquisa}
        \cvevent{Pós-Graduação}{\textbf{| Laboratório de Otimização e Combinatória}}{Mar. 2019 - Atual}{Campinas - SP}
	    {Desenvolvimento de modelos, relaxações lagrangianas e heurísticas para problemas combinatórios NP-difíceis, incluindo (mas não limitado a):} \\
	      \begin{itemize}
		        \item Roteamento de veículos e geração de Aŕvores de \textit{Multicast}.
		  \end{itemize}
		\divider
		\cvevent{Graduação}{\textbf{| Núcleo de Estudos em Aprendizado de Máquina e Otimização}}{Fev. 2017 - Dez. 2021}{Russas - CE}{}
        \begin{itemize}
        	\item Modelagem e desenvolvimento de relaxações e heurísticas para o Problema de Máxima Diversidade de Grupos.
        	\item Desenvolvimento de heurísticas e cortes de otimalidade para o problema de corte máximo em grafos.
        	\item Estudo de ciência dos dados e algoritmos de Aprendizado de Máquina, gerando resultados aplicados em competições na plataforma Kaggle.
        \end{itemize}
% 	 Experiência com Programação Linear Inteira, métodos heurísticos e desenvolvimento/análise de algoritmos. \\
% 	 Experiência com resolvedores comerciais ILOG CPLEX, Gurobi e AMPL. \\
% 	 Experiência com criação de instâncias com auxílio de simuladores de tráfego e rede. \\
% 	 Análise estatística de resultados utilizando Linguagem Python e R. \\

        % ----- EXPERIENCE -----
        \cvsection{Monitoria e Docência}
        \medskip
        \cvevent{Professor Assistente}{\textbf{| Universidade Estadual de Campinas}}{Jan. 2020 - Dez. 2020}{Campinas - SP}
        {Professor Assistente na disciplina de Desafios de Programação - MO521 e Introdução a Algoritmos e Programação - MO102. \\}
        \divider
        % \medskip
        % % \cvevent{Professor Assistente}{\textbf{| Universidade Estadual de Campinas}}{Ago. 2020 - Dez. 2020}{Campinas - SP}
        % % {Professor Assistente na disciplina de Introdução a Algoritmos e Programação - MO102. \\}
        % % \divider
        % % \medskip
        % \cvevent{Monitor}{\textbf{| Universidade Federal do Ceará}}{Jan. 2017 - Dez. 2017}{Russas - CE}
        % {Monitor da Disciplina de Fundamentos de Programação. \\}
        % \divider
    \end{paracol}
    
    \begin{paracol}{1}
    % ----- EDUCATION -----
        \cvsection{Educação}
            \medskip
            \cvevent{\textbf{Doutor em Ciência da Computação}}{| Otimização Combinatória}{Mar. 2021 - Mar. 2025}{Universidade Estadual de Campinas}
            \medskip
		      \begin{itemize}
			      \item{Orientadores: Prof. Dr. Fábio L. Usberti e Dr. Rafael K. Arakaki.}
			      \item{Cursos: \textcolor{SecondaryColor}{Programação Paralela}, \textcolor{SecondaryColor}{Complexidade de Algoritmos} e \textcolor{SecondaryColor}{Algoritmos de Aproximação}.}
		      \end{itemize}
            \divider
            \cvevent{\textbf{Mestre em Ciência da Computação}}{| Otimização Combinatória}{Mar. 2019 - Mar. 2021}{Universidade Estadual de Campinas}
            \medskip
		      \begin{itemize}
			      \item{Orientadores: Prof. Dr. Fábio L. Usberti e Prof. Dr. Cid C. de Souza.}
			      \item{Dissertação: Formulações e Heurísticas para o Problema de Máximo Atendimento em Roteamento Multicast com Restrições de QoS.}
			      \item{Cursos: \textcolor{SecondaryColor}{Algoritmos em Grafos}, \textcolor{SecondaryColor}{Programação Linear e Inteira} e \textcolor{SecondaryColor}{ Tópicos em Otimização Combinatória}.}
			      \item{Uma cópia da minha dissertação está disponível nesse
			      \href{https://hdl.handle.net/20.500.12733/1641777}{\textcolor{ThirdColor}{link}}}
		      \end{itemize}
            \divider
            \cvevent{\textbf{Bacharel em Ciência da Computação}}{}{Mar. 2015 - Dez. 2018}{Universidade Federal do Ceará}
            % \medskip
		      \begin{itemize}
		          \item{Orientador: Prof. Dr. Pablo L. B. Soares.}
			      \item{TCC: Utilização de desigualdades válidas baseadas em condições de otimalidade na construção de algoritmos heurísticos para o problema do corte máximo.}
  			      \item{Uma cópia do meu Trabalho de Conclusão de Curso está disponível nesse
			      \href{http://www.repositorio.ufc.br/handle/riufc/39085}{\textcolor{ThirdColor}{link}}}
		      \end{itemize}
		      
        % ----- EDUCATION -----
        \cvsection{Publicações}
        \medskip
    	    \textbf{Araújo, C. V. D.}; Usberti, F. L.; de Souza, C. C.
    	    \textcolor{SecondaryColor}{Lagrangian Relaxation to the Problem of Maximum Service in Multicast Routing with QoS constraints}. International Transactions in Operational Research (ITOR), 2022. Vol. 0 p. 0-0 (\emph{Em revisão}) \\
    	    \divider \\
    		\textbf{Araújo, C. V. D.}; Soares, P. L. B.
    	    \textcolor{SecondaryColor}{Genetic Algorithms with Optimality Cuts to the Max-Cut Problem}. Special Issue Bio-inspired Computing Emerging Theories and Industry Applications (VSI-bioc), 2020. Vol. 0 p. 0-0 (\emph{Em revisão}) \\
    	    \divider \\
    	    \textbf{Araújo, C. V. D.}; Figueiredo, T. F.
    	    \textcolor{SecondaryColor}{O Problema Da Diversidade Máxima de Grupos: uma abordagem de programação linear inteira}.  L Simpósio Brasileiro de Pesquisa Operacional (SBPO), 2018. Vol. 0 p 0-0 \\
    		\divider \\
    		\textbf{Araújo, C. V. D.}; Figueiredo, T. F.
    		\textcolor{SecondaryColor}{Relaxação Lagrangiana Aplicada ao Problema da Diversidade Máxima de Grupos}. Encontros universitários - UFC, 2018. Vol. 0 p. 0-0 (\emph{Resumo}) \\
    		\divider \\
    		\textbf{Araújo, C. V. D.}; Soares, P. L. B. 
    		\textcolor{SecondaryColor}{Algoritmo Genético para o Problema Do Corte Máximo}. Encontros universitários - UFC, 2018. Vol. 0 p. 0-0 (\emph{Resumo}) \\
            \divider \\
            \textbf{Araújo, C. V. D.}; Soares, P. L. B.
            \textcolor{SecondaryColor}{Estudo de Abordagens para o problema de Corte Máximo}. Encontros universitários  - UFC, 2017.Vol. 0 p. 0-0 (\emph{Resumo}) \\
	    \medskip
        % ----- PROJECTS -----
        \cvsection{Honras e Prêmios}
        \begin{itemize}
            \item Melhor trabalho da categoria Iniciação Científica nos \textcolor{SecondaryColor}{Encontros Universitários, 2018}.
            \item Segundo lugar na etapa regional do \textcolor{SecondaryColor}{\textit{International Collegiate Programming Contest} (ICPC), 2018}.
            \item Vencedor da etapa nacional da \textcolor{SecondaryColor}{Olimpíada de Língua Portuguesa - Escrevendo o Futuro no ano de 2008}.     
        \end{itemize}
    \end{paracol}
\end{document}
